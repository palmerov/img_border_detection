% Artículo científico: vectorización del kernel de detección de bordes (auto, guiado, explícito)
% Análisis con Intel Advisor — Estilo Jornadas Sarteco (español)

\documentclass[twocolumn,twoside]{Jornadas}
\usepackage[utf8]{inputenc}
\usepackage[spanish]{babel}
\usepackage{textcomp}
\usepackage{amsmath}
\usepackage{listings}
\usepackage{algorithm}
\usepackage{algorithmicx}
\usepackage{algcompatible}
\usepackage{adjustbox}
\usepackage{graphicx}
\usepackage{color}
\usepackage{caption}
\captionsetup{font=footnotesize}
\usepackage[caption=false,font=footnotesize]{subfig}
\usepackage{placeins}
\usepackage{hyperref}
\usepackage{url}
\usepackage{booktabs}

\floatname{algorithm}{Algoritmo}

\setcounter{topnumber}{10}
\setcounter{bottomnumber}{10}
\setcounter{totalnumber}{10}
\renewcommand{\topfraction}{1}
\renewcommand{\bottomfraction}{1}
\renewcommand{\textfraction}{0}
\renewcommand{\floatpagefraction}{1}

\definecolor{gray97}{gray}{.97}
\definecolor{gray75}{gray}{.75}
\definecolor{gray45}{gray}{.45}

\lstset{
     inputencoding=utf8,
     extendedchars=true,
     backgroundcolor=\color{gray97},
     stringstyle=\ttfamily,
     showstringspaces = false,
     basicstyle=\scriptsize\ttfamily,
     commentstyle=\color{gray45},
     keywordstyle=\bfseries,
     linewidth=.98\columnwidth,
     xleftmargin=3mm,
     breaklines=true,
     numbers=left,
     numbersep=6pt,
     numberstyle=\tiny,
     breaklines=true,
}

\def\BibTeX{{\rm B\kern-.05em{\sc i\kern-.025em b}\kern-.08em
    T\kern-.1667em\lower.7ex\hbox{E}\kern-.125emX}}

\graphicspath{{.}{./Figuras/}}

\begin{document}

\title{Análisis de rendimiento de la vectorización automática, guiada y explícita en un kernel de detección de bordes mediante Intel Advisor}

\author{%
     Juan Pablo Palmero Valdés%
     \thanks{Lic. en Ciencias de la Computación. Benemérita Universidad Autónoma de Puebla. E-mail: {\tt palmerovaldes99@gmail.com}.},
     Mario Rossainz López%
     \thanks{Doctor en Ciencias de la Computación. Benemérita Universidad Autónoma de Puebla. E-mail: {\tt mario.rossainz@correo.buap.mx}.}
}

\maketitle
\markboth{}{}
\pagestyle{empty}
\thispagestyle{empty}

\begin{abstract}
La detección de bordes es un kernel numérico fundamental en visión por computador y procesamiento de imágenes. En este trabajo se implementa el detector de bordes por magnitud del gradiente \(G[i] = \sqrt{G_x[i]^2 + G_y[i]^2}\) en C++ moderno en cuatro variantes: línea base escalar, vectorización automática del compilador, vectorización guiada (OpenMP SIMD) y vectorización explícita (Intel AVX2). Se analiza el rendimiento y la eficiencia de vectorización con Intel Advisor, reportando tiempos de ejecución y porcentajes de vectorización. Las tres variantes vectorizadas (auto, guiada y explícita) alcanzan 100\% de tiempo en bucle vectorizado según Intel Advisor; la explícita (AVX2) es la más rápida en ejecución directa (aprox. 2,2\(\times\) frente a la escalar) y la guiada la sigue.
\end{abstract}

\begin{keywords}
Detección de bordes, magnitud del gradiente, kernel Sobel, SIMD, vectorización automática, OpenMP SIMD, Intel AVX2, Intel Advisor, análisis de rendimiento.
\end{keywords}

%!TEX root = main.tex
\section{Introducción}
\PARstart{L}{a} detección de bordes es una operación fundamental en visión por computador, procesamiento de imágenes médicas, percepción robótica y como etapa de preprocesado para redes neuronales convolucionales. Los bordes marcan cambios bruscos de intensidad que suelen corresponder a contornos de objetos y permiten reducir la imagen a información estructural: formas, contornos y regiones. Desde el punto de vista computacional, los bordes se caracterizan por una magnitud del gradiente elevada, por lo que los enfoques estándar utilizan operadores como Sobel, Prewitt o Canny~\cite{canny1986}.

El kernel numérico en el que nos centramos es el cómputo de la magnitud del gradiente: en cada píxel \(i\), la fuerza del borde viene dada por
\begin{equation}
G[i] = \sqrt{G_x[i]^2 + G_y[i]^2},
\end{equation}
donde \(G_x\) y \(G_y\) son los gradientes horizontal y vertical obtenidos convolucionando la imagen con kernels pequeños (p.ej.\ Sobel 3\(\times\)3). Este kernel está limitado por cómputo y es muy regular, lo que lo convierte en un buen candidato para la vectorización SIMD en CPUs modernas.

El objetivo de este trabajo es comparar tres estrategias de vectorización---automática (solo compilador), guiada (directivas como OpenMP SIMD) y explícita (intrinsics SIMD escritas a mano)---sobre una implementación en C++ de este kernel, y analizar su rendimiento y eficiencia de vectorización con Intel Advisor. Implementamos una línea base escalar más las tres versiones vectorizadas, las ejecutamos bajo Intel Advisor para obtener tiempos y métricas de vectorización, y presentamos los resultados en este artículo. El documento se organiza así: la Sección~\ref{sec:background} repasa el kernel y los conceptos de vectorización; la Sección~\ref{sec:method} describe la implementación y la metodología basada en Advisor; la Sección~\ref{sec:experimental} presenta los resultados experimentales; la Sección~\ref{sec:conclusions} resume las conclusiones y líneas futuras.

%!TEX root = main.tex
\section{Background and related work}
\label{sec:background}

\subsection{Gradient-magnitude edge detection}

An edge is defined as a strong local change in image intensity. For a discrete 2D image, the gradient is approximated by convolving the image with derivative kernels. The Sobel operator uses two 3\(\times\)3 kernels to compute \(G_x\) (horizontal gradient) and \(G_y\) (vertical gradient). The magnitude \(G\) in Eq.~(1) is then a scalar measure of edge strength at each pixel. Convolution is implemented as a double loop over the image with a 3\(\times\)3 neighbourhood; the magnitude step is a single pass over the pixels. Both phases are data-parallel and exhibit regular memory access, which is favourable for SIMD and for compiler auto-vectorization~\cite{simd-survey}.

\subsection{Vectorization strategies}

\textbf{Auto-vectorization:} The compiler (e.g.\ Intel oneAPI \texttt{icpx} or GCC with \texttt{-O3 -march=native}) attempts to map scalar operations to SIMD instructions without source changes. Effectiveness depends on loop structure, alignment, and absence of dependencies that prevent vectorization.

\textbf{Guided (implicit) vectorization:} The programmer adds directives such as \texttt{\#pragma omp simd} to mark loops as vectorisable, helping the compiler and optionally enabling OpenMP SIMD-specific optimisations~\cite{openmp-simd}.

\textbf{Explicit vectorization:} The programmer uses SIMD intrinsics (e.g.\ Intel AVX/AVX2 \texttt{\_mm256\_*} for 256-bit vectors) to implement the kernel by hand~\cite{intel-intrinsics}. This gives full control and can achieve high utilisation when the compiler or guided approach falls short, at the cost of portability and maintainability.

Intel Advisor~\cite{inteladvisor} provides roofline-style analysis, vectorisation advice, and metrics such as vectorisation coverage and loop body efficiency, which we use to interpret the performance of each variant.

%!TEX root = main.tex
\section{Metodología}
\label{sec:method}

\subsection{Implementación escalar}

Implementamos el kernel de detección de bordes en C++ moderno (C++17). La versión escalar consta de:
\begin{enumerate}
\item Carga de una imagen en escala de grises (formato PGM) en un buffer contiguo de \texttt{double}.
\item Convolución de la imagen con los kernels de Sobel \(G_x\) y \(G_y\) 3\(\times\)3 con salida de mismo tamaño (los píxeles de borde se omiten o replican). Dos bucles anidados sobre filas y columnas calculan cada píxel de salida con una ventana 3\(\times\)3 (esténcil de 9 puntos).
\item Cálculo de la magnitud del gradiente \(G[i] = \sqrt{G_x[i]^2 + G_y[i]^2}\) en un único bucle sobre todos los píxeles.
\end{enumerate}
No se usan directivas SIMD ni intrinsics; sirve como línea base y como referencia de ``sin vectorización'' para Intel Advisor.

El Algoritmo~\ref{alg:scalar-loop} detalla el bucle principal de la convolución (un píxel por iteración) y el bucle de magnitud. En la variante escalar ambos bucles son puramente escalares; en las variantes guiada y explícita estos mismos bucles son el objetivo de la vectorización.

\begin{algorithm}[htb]
\caption{Bucle escalar: convolución y magnitud}
\label{alg:scalar-loop}
\begin{algorithmic}[1]
\Require Imagen \(I\), kernels \(K_x\), \(K_y\) 3\(\times\)3; dimensiones \(H\), \(W\)
\Ensure Imagen de bordes \(G\)
\State Inicializar \(G_x\), \(G_y\) del tamaño de \(I\)
\For{\(y \gets 1\) \textbf{to} \(H-2\)}
\For{\(x \gets 1\) \textbf{to} \(W-2\)}
\State \(gx \gets 0\); \(gy \gets 0\)
\For{\(dy \gets -1\) \textbf{to} \(1\)}
\For{\(dx \gets -1\) \textbf{to} \(1\)}
\State \(gx \gets gx + I[y+dy,x+dx] \cdot K_x[dy+1,dx+1]\)
\State \(gy \gets gy + I[y+dy,x+dx] \cdot K_y[dy+1,dx+1]\)
\EndFor
\EndFor
\State \(G_x[y,x] \gets gx\); \(G_y[y,x] \gets gy\)
\EndFor
\EndFor
\For{cada píxel \(i\) en el interior}
\State \(G[i] \gets \sqrt{G_x[i]^2 + G_y[i]^2}\)
\EndFor
\State \Return \(G\)
\end{algorithmic}
\end{algorithm}

\subsection{Versión auto-vectorizada}

El mismo código fuente se compila con optimización agresiva y flags específicos de arquitectura: \texttt{-O3 -march=native} (con Intel oneAPI se usa \texttt{icpx}). No se añaden pragmas ni intrinsics. Se espera que el compilador auto-vectorice los bucles de convolución y magnitud cuando sea legal. Esta variante se denomina ``auto'' en los resultados.

\subsection{Vectorización guiada (implícita)}

Una segunda variante se construye con la misma estructura algorítmica pero con \texttt{\#pragma omp simd} aplicado al bucle interno de la convolución y al bucle de magnitud. El proyecto se enlaza con OpenMP (\texttt{-fiopenmp} en Intel oneAPI). Así se guía al compilador para generar código SIMD en esos bucles y Advisor puede reportar la vectorización guiada.

\subsection{Vectorización explícita}

Una tercera variante implementa la convolución y el cómputo de la magnitud usando intrinsics Intel AVX2 (\texttt{\_mm256\_*}). Se procesan cuatro valores \texttt{double} por iteración (registros de 256 bits). El bucle interno de la convolución se desenrolla para calcular cuatro píxeles consecutivos cada vez; el resto se trata con un epílogo escalar. La etapa de magnitud usa \texttt{\_mm256\_loadu\_pd}, \texttt{\_mm256\_mul\_pd}, \texttt{\_mm256\_add\_pd} y \texttt{\_mm256\_sqrt\_pd}, con cola escalar. Esta variante se compila con \texttt{-O3 -march=native} para disponer de FMA y otras instrucciones.

\subsection{Análisis de rendimiento con Intel Advisor}

La compilación se realiza con \texttt{icpx} y los parámetros necesarios en cada caso: escalar sin auto-vectorización; auto con \texttt{-O3 -march=native}; guiada con \texttt{-O3 -march=native -fiopenmp}; explícita con \texttt{-O3 -march=native}. Para cada variante se ejecutó el binario bajo Intel Advisor (Survey) sobre una imagen de entrada fija y se utilizaron los informes de vectorización para obtener métricas como porcentaje de código vectorizado y eficiencia del bucle. Las imágenes de entrada son PGM en escala de grises; el tamaño y la ruta se mantienen fijos en todas las ejecuciones para que tiempos y métricas sean comparables.

%!TEX root = main.tex
\section{Resultados experimentales}
\label{sec:experimental}

Los experimentos se ejecutaron en una máquina con \textbf{1 hilo de CPU} y conjunto de instrucciones \textbf{AVX}. La imagen de entrada utilizada en todas las ejecuciones es de \textbf{256\(\times\)256} píxeles (PGM en escala de grises). Los tiempos de la Tabla~\ref{tab:time} se obtuvieron por \textbf{ejecución directa} de cada binario (cronómetro alrededor del kernel, mismo criterio en las cuatro variantes). Adicionalmente, cada variante se ejecutó bajo \textbf{Intel Advisor} (Survey) para obtener las métricas de vectorización y las capturas de la subsección~\ref{sec:advisor-figs}.

\subsection{Imágenes procesadas}

La Figura~\ref{fig:processed} muestra la imagen de entrada original y el resultado de la detección de bordes producido por la variante explícita (AVX2). La magnitud del gradiente destaca claramente los contornos principales. Salidas adicionales para las variantes auto, guiada y explícita, así como otras imágenes de prueba (p.ej.\ \texttt{guitar.pgm}, \texttt{bee.pgm}), están disponibles en la carpeta \texttt{img/} del repositorio.

\begin{figure}[htb]
\begin{center}
\subfloat[Entrada original (256\(\times\)256)]{\includegraphics[width=0.48\columnwidth]{input}}
\hfill
\subfloat[Resultado detección de bordes (AVX2 explícito)]{\includegraphics[width=0.48\columnwidth]{output_explicit}}
\end{center}
\caption{Ejemplo: imagen original y salida de detección de bordes (variante explícita).}
\label{fig:processed}
\end{figure}

\subsection{Tiempo de ejecución}

La Tabla~\ref{tab:time} recoge el tiempo de ejecución en \textbf{milisegundos (ms)} para la línea base escalar y las tres versiones vectorizadas. Todas las medidas se tomaron por ejecución directa (\texttt{high\_resolution\_clock} alrededor del kernel).

\begin{table}[htb]
\caption{Tiempo de ejecución por variante (ejecución directa, en ms).}
\label{tab:time}
\begin{center}
{\footnotesize
\begin{tabular}{lc}
\toprule
Variante   & Tiempo (ms) \\
\midrule
Escalar   & 0,52 \\
Auto     & 0,34 \\
Guiada (OpenMP SIMD) & 0,27 \\
Explícita (AVX2)      & 0,24 \\
\bottomrule
\end{tabular}
}
\end{center}
\end{table}

La versión escalar tarda 0,52\,ms; la auto, 0,34\,ms; la guiada, 0,27\,ms; la explícita, 0,24\,ms. Las variantes vectorizadas son más rápidas que la escalar; la explícita (AVX2) es la más rápida y mejora respecto de la guiada.

\subsection{Métricas de vectorización de Intel Advisor}

Se utilizó Intel Advisor para recoger datos de Survey en cada compilación. La Tabla~\ref{tab:advisor} resume las métricas de vectorización según el resumen ``Vectorization and Code Insights'': tiempo de CPU en bucles vectorizados e instrucciones AVX.

\begin{table}[htb]
\caption{Métricas de vectorización de Intel Advisor (tiempo en código vectorizado).}
\label{tab:advisor}
\begin{center}
{\footnotesize
\begin{tabular}{lc}
\toprule
Variante   & Vectorización (tiempo en bucle vectorizado) \\
\midrule
Escalar   & 0\% \\
Auto     & 100\% \\
Guiada   & 100\% \\
Explícita & 100\% \\
\bottomrule
\end{tabular}
}
\end{center}
\end{table}

Escalar: 0\% en código vectorizado; auto, guiada y explícita: 100\% en un bucle vectorizado (AVX).

\subsection{Capturas de Intel Advisor}
\label{sec:advisor-figs}

Las Figuras~\ref{fig:advisor-auto}--\ref{fig:advisor-explicit} muestran el resumen de Advisor para cada variante.

\begin{figure}[htb]
\begin{center}
\includegraphics[width=\columnwidth]{auto_general}
\end{center}
\caption{Resumen Intel Advisor: variante auto-vectorizada (\texttt{bdet\_auto}).}
\label{fig:advisor-auto}
\end{figure}

\begin{figure}[htb]
\begin{center}
\includegraphics[width=\columnwidth]{guided_general}
\end{center}
\caption{Resumen Intel Advisor: variante guiada (OpenMP SIMD) (\texttt{bdet\_guided}).}
\label{fig:advisor-guided}
\end{figure}

\begin{figure}[htb]
\begin{center}
\includegraphics[width=\columnwidth]{exp_general}
\end{center}
\caption{Resumen Intel Advisor: variante explícita (AVX2) (\texttt{bdet\_explicit}).}
\label{fig:advisor-explicit}
\end{figure}

\subsection{Aceleración y comparación}

\begin{itemize}
\item Tiempos por ejecución directa (ms): escalar 0,52; auto 0,34; guiada 0,27; explícita 0,24.
\item Aceleración frente a escalar: auto 1,5\(\times\); guiada 1,9\(\times\); explícita 2,2\(\times\).
\item Mejora de la explícita respecto de la guiada: aproximadamente 1,1\(\times\) (0,27\,ms frente a 0,24\,ms).
\end{itemize}

Las métricas de Advisor (Tabla~\ref{tab:advisor}) coinciden con las figuras. Las diferencias de tiempo entre auto, guiada y explícita reflejan el mayor control de la guiada y la explícita frente a la auto-vectorización.

%!TEX root = main.tex
\section{Conclusiones}
\label{sec:conclusions}

En este trabajo se ha implementado el kernel de detección de bordes por magnitud del gradiente \(G[i] = \sqrt{G_x[i]^2 + G_y[i]^2}\) en C++ moderno en cuatro variantes: línea base escalar, vectorización automática del compilador, vectorización guiada (OpenMP SIMD) y vectorización explícita (Intel AVX2). Se ha analizado el rendimiento y la eficiencia de vectorización con Intel Advisor.

Los tiempos de ejecución directa (en ms) son: escalar 0,52; auto 0,34; guiada 0,27; explícita 0,24. La variante explícita (AVX2) es la más rápida, con una aceleración de aproximadamente 2,2\(\times\) frente a la escalar y 1,1\(\times\) frente a la guiada. Con Intel Advisor se confirmó que las tres variantes vectorizadas (auto, guiada y explícita) alcanzan 100\% de tiempo en un bucle vectorizado (AVX); solo la escalar queda en 0\%.

Advisor reporta 0\% de vectorización para escalar y 100\% para las tres variantes vectorizadas. Las diferencias de rendimiento entre estas últimas se deben al mayor grado de optimización de la guiada y la explícita frente a la auto-vectorización.

En resumen, las tres variantes vectorizadas (auto, guiada y explícita) mejoran frente a la línea base escalar; la explícita AVX2 es la más rápida y ofrece el máximo control a costa de portabilidad. Como trabajo futuro se podría considerar el uso de imágenes de mayor tamaño, otros kernels (p.ej.\ Canny completo) o la comparación con implementaciones en GPU.


\section*{Agradecimientos}

Agradecemos a la Benemérita Universidad Autónoma de Puebla (BUAP) y al CONACYT por el apoyo institucional. Asimismo, nuestro agradecimiento a los profesores de la Facultad de Ciencias de la Computación de la BUAP por su orientación y apoyo durante la realización de este trabajo.

\nocite{*}
\bibliographystyle{Jornadas}
\bibliography{biblio}

\end{document}
