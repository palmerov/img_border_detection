%!TEX root = main.tex
\section{Experimental results}
\label{sec:experimental}

Experiments were run on \textbf{[PLACEHOLDER: machine description, e.g. CPU model, number of cores, RAM]}. The input image used for all runs is \textbf{[PLACEHOLDER: image size, e.g. 1024\(\times\)1024 or 4096\(\times\)4096]} pixels. Each configuration was executed multiple times; the values below correspond to \textbf{[PLACEHOLDER: e.g. median of 5 runs / average of 10 runs]}.

\subsection{Execution time}

Table~\ref{tab:time} reports the execution time (in \textbf{[PLACEHOLDER: units, e.g. milliseconds (ms) or microseconds (\(\mu\)s)]}) for the scalar baseline and the three vectorized versions. \textbf{Placeholder values are used; replace with actual measurements.}

\begin{table}[htb]
\caption{Execution time per variant. \textbf{[REPLACE T1--T4 with measured values.]}}
\label{tab:time}
\begin{center}
{\footnotesize
\begin{tabular}{lcc}
\toprule
Variant   & Time & Unit \\
\midrule
Scalar   & T1 & \textbf{[e.g. ms]} \\
Auto     & T2 & \textbf{[e.g. ms]} \\
Guided (OpenMP SIMD) & T3 & \textbf{[e.g. ms]} \\
Explicit (AVX2)      & T4 & \textbf{[e.g. ms]} \\
\bottomrule
\end{tabular}
}
\end{center}
\end{table}

Summary: execution time for the scalar version is T1; for the auto-vectorized version, T2; for the guided (implicit) version, T3; and for the explicit (AVX2) version, T4. \textbf{[PLACEHOLDER: Once data are available, add one or two sentences comparing T1--T4, e.g. ``The explicit version is the fastest, with a X.X\(\times\) speedup over scalar.''.]}

\subsection{Intel Advisor vectorization metrics}

Intel Advisor was used to collect survey and, where applicable, roofline data for each build. Table~\ref{tab:advisor} summarises the vectorization-related metrics. \textbf{Replace placeholder labels with the actual metrics reported by Advisor (e.g. ``Vectorization \%'', ``Loop body efficiency'', ``Estimated potential gain'').}

\begin{table}[htb]
\caption{Intel Advisor vectorization metrics. \textbf{[REPLACE P1--P4 and any other columns with real metric names and values.]}}
\label{tab:advisor}
\begin{center}
{\footnotesize
\begin{tabular}{lcc}
\toprule
Variant   & Vectorization percent & \textbf{[Other metric]} \\
\midrule
Scalar   & P1 & \textbf{[--]} \\
Auto     & P2 & \textbf{[--]} \\
Guided   & P3 & \textbf{[--]} \\
Explicit & P4 & \textbf{[--]} \\
\bottomrule
\end{tabular}
}
\end{center}
\end{table}

Vectorization percent for the scalar version is P1 (expected 0\% or N/A); for the auto-vectorized version, P2; for the guided version, P3; and for the explicit version, P4. \textbf{[PLACEHOLDER: Add a short interpretation, e.g. ``Advisor reports that the magnitude loop is fully vectorised in the explicit build, whereas the auto build only vectorises it partially.''.]}

\subsection{Speedup and comparison}

\textbf{[PLACEHOLDER: Fill with computed speedups once T1--T4 are known.]}
\begin{itemize}
\item Speedup of auto vs scalar: \textbf{[SPEEDUP\_AUTO = T1/T2]}.
\item Speedup of guided vs scalar: \textbf{[SPEEDUP\_GUIDED = T1/T3]}.
\item Speedup of explicit vs scalar: \textbf{[SPEEDUP\_EXPLICIT = T1/T4]}.
\item Relative improvement of explicit with respect to guided (implicit): \textbf{``...a improvement of X.X of the explicit version with respect to the implicit (guided) version...''} (e.g. 1.2\(\times\) faster, or ``X.X\% reduction in time'').
\end{itemize}

\textbf{[PLACEHOLDER: If you have roofline or other Advisor figures, add them here and reference as Fig.~\ref{fig:roofline} or similar.]}

%\begin{figure}[htb]
%\begin{center}
%  \includegraphics[width=\columnwidth]{roofline_placeholder}
%\end{center}
%\caption{Roofline model (Intel Advisor). \textbf{[Replace with actual figure.]}}
%\label{fig:roofline}
%\end{figure}
